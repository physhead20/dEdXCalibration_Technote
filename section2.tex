\section{Data Samples}\label{sec:DataSamples}
The data samples used for this calibration span Run-I and Run-II. The details of all the data samples are outlined in \href{https://docs.google.com/spreadsheets/d/1_0kNCKBIIx53f6vopqN2OijtcTICHD9rDvN_YKGH2mI/edit?usp=sharing}{this data production spreadsheet} and summarized in Table \ref{tab:samples}. It is worth noting, that in the inclusive pion cross-section analysis (for which a good deal of this calibration is driven by), the ``High-Yield'' beamline reconstruction is what is used to increase the statistics of the sample. Since the ``Picky Track'' reconstruction used here represents only a subset of that data, we don't use the entire measurement sample to tune our response. 

The details of the beamline reconstruction and the mass filter are provided in the next section.

\begin{center}
\begin{table}[htb]
	\begin{center}
	%\resizebox{0.45\textwidth}{!}{%
	\begin{tabular}{c|c|c|c|c}
	\multicolumn{4}{c}{\textbf{Summary of Samples}} \\
	\hline \hline
	 Run Period & LArIATsoft Version & Beamline Reconstruction & Mass Filter & Calibration/Validation\\
	\hline
	Run-I & \verb!v06_34_01! & Negative Polarity Picky Track & $\pi, \mu, e$ & Calibration \\
	\hline
	Run-I & \verb!v06_34_01! & Positive Polarity Picky Track & $\pi, \mu, e$ & Validation \\
	\hline
	Run-I & \verb!v06_34_01! & Positive Polarity Picky Track & Proton & Validation \\
	\hline
	Run-II & \verb!v06_34_01! & Positive Polarity Picky Track & $\pi, \mu, e$ & Calibration \\
	\hline
	Run-II & \verb!v06_34_01! & Negative Polarity Picky Track & $\pi, \mu, e$ & Validation \\
	\hline
	Run-II & \verb!v06_34_01! & Positive Polarity Picky Track & Proton & Validation \\
	\hline
	\end{tabular}%}
	\caption{Summary of the data samples used for the calibration study. A sample listed as a ``calibration'' sample was used, tuned, checked, and re-tuned until the desired calibration was achieved. A sample listed at ``validation'' was then checked using the calorimetry constants derived for that run period. }
	\label{tab:samples}
	\end{center}
\end{table}
\end{center}

The corresponding data driven Monte Carlo (DDMC) samples which match the data samples listed in Table \ref{tab:samples} are used to tune the MC calibration constants and can be found on the data spreadsheet listed above.



\subsection{Event Selection}\label{sec:EventSelection}
For each sample, we outline the event selection used to select this data.

\begin{itemize}
\item \textbf{Time Stamp Filter}

A filter is used to select events which occured in time with the beam. These events typically coincide with the first 6 seconds of the beam spill, and therefore the events are filtered using the following LArIATsoft settings

\begin{verbatim}
tfilt:      @local::lariat_timestampfilter

# ====================================================================
# Specify range of events to select.  For Run I/II:
#   - pedestal events:  ~ 0.  - 1.2 sec
#   - beam events:      ~ 1.2 - 5.5 sec
#   - cosmic events:    ~ > 5.5 sec
#   (default selects ALL events)
physics.filters.tfilt.T1:                       1.2
physics.filters.tfilt.T2:                       5.5
physics.filters.tfilt.RequireRawDigits:         true

\end{verbatim}



\item \textbf{Beamline Reconstruction}

The standard LArIAT beamline reconstruction is used to select events which have a wire chamber track and TOF information in an individual event using the following modules.
\begin{verbatim}
### beamline elements ###

wctrack:     @local::lariat_wctrackbuilder
tof:         @local::lariat_tof
agcounter:   @local::lariat_aerogel
\end{verbatim}


\textbf{For Run-I we use these default parameters:}
\begin{verbatim} 
physics.producers.wctrack.PickyTracks:                          true
physics.producers.tof.HitThreshold:                           -10.0  
physics.producers.tof.HitDiffMeanUS:                            0.6  
physics.producers.tof.HitDiffMeanDS:                            1.0  
physics.producers.tof.HitMatchThresholdUS:                      3.0  
physics.producers.tof.HitMatchThresholdDS:                      6.0  
physics.producers.tof.HitWait:                                  20.
\end{verbatim}

\textbf{For Run-II we use these default parameters:}
\begin{verbatim} 
physics.producers.wctrack.PickyTracks:                          true
physics.producers.tof.HitThreshold:                             -3.
physics.producers.tof.HitDiffMeanUS:                            0.5  
physics.producers.tof.HitDiffMeanDS:                            0.4  
physics.producers.tof.HitMatchThresholdUS:                      3.0  
physics.producers.tof.HitMatchThresholdDS:                      6.0  
physics.producers.tof.HitWait:                                  20.
\end{verbatim}

\item \textbf{Particle Mass Filtering}
Using the beamline reconstruction, it is possible to calculate the mass of a given track using the following equation

\begin{equation}
mass = \frac{p}{c}\sqrt{(\frac{TOF \times c}{l})^2 -1}
\end{equation}
where $p$ represents the measured momentum from the wire chamber, $TOF$ represents the time-of-flight measured as the difference between the two time-of-flight paddles in the LArIAT beamline, $l$ is path length the particle traveled down the beamline, and $c$ represents the speed of light.

Using this it is possible to plot the mass of each track reconstructed in the particle beamline, as shown in Figure \ref{fig:mass}. The classification of events into the different samples follows:

\begin{itemize}

\item $\pi, \mu, e$: 0~MeV $<$ mass $<$ 350~MeV

\item kaon: 350~MeV $<$ mass $<$ 650~MeV

\item proton: 650~MeV $<$ mass $<$ 3000~MeV

\end{itemize}

\begin{figure}[htb]
\centering
\includegraphics[width=0.70\textwidth]{images/mass.png}
\caption{The mass plotted for a sample of Run-II events reconstructed in the beamline. The classification of the events into $\pi, \mu, e$, kaon, or proton is based on this distribution.}
\label{fig:mass}
\end{figure}

\item \textbf{LArTPC Reconstruction}

Finally, events are reconstructed inside the TPC and we select events that satisfy the following requirements

\begin{itemize}
\item At least one track reconstructed inside the TPC with length $>$~10~cm
\item The event has fewer than three tracks with a length less than 5~cm reconstructed (shower veto)
\item One and only one TPC track can be matched to a wire chamber track (-4.0~cm$< \Delta X <$ 6.0~cm, -5.0~cm$< \Delta Y <$ 5.0~cm, $\alpha < 10$ degrees)

\end{itemize}


\end{itemize}

Events passing all these selection requirements are used for the dE/dX calibration. The event reduction table for all these cuts is provided in Section \ref{sec:Results} for each relevant sub-sample as well as the results from the calorimetry constant tuning.
