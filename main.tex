\documentclass[a4paper]{article}

%\usepackage[english]{babel}
%\usepackage[utf8]{inputenc}
\usepackage{graphicx}
\usepackage{epsfig}
%\usepackage{amsmath}
\usepackage{graphicx}
\usepackage[colorinlistoftodos]{todonotes}
\usepackage{a4}
%\usepackage{amssymb}
\usepackage{color}
\usepackage{lineno}
\usepackage{ulem}
\usepackage{enumerate}
\usepackage{comment}

\usepackage[left=2.5cm,right=2cm,top=2.5cm,bottom=2.cm]{geometry} 

%% for long url reference
\usepackage{hyperref}
\usepackage{url}
\makeatletter
\def\url@mystyle{%
  \@ifundefined{selectfont}{\def\UrlFont{\sf}}{\def\UrlFont{\small\ttfamily}}}
\makeatother
\urlstyle{my}



\renewcommand{\thefootnote}{\alph{footnote}}
\renewcommand{\topfraction}{.99}
\renewcommand{\bottomfraction}{.99}

\title{Momentum based dE/dX Calibration of LArIAT for Run I and Run II}

%%%%%%%%%%%%%%%%%%%%%%%%%%%%%%%%%
\begin{document}
%%%%%%%%%%%%%%%%%%%%%%%%%%%%%%%%%
\def\Journal#1#2#3#4{{#1} {\bf #2}, #3 (#4)}
\def\etal{{\it et\ al.}}
\def\numunue{\nu_\mu\rightarrow\nu_e}
\def\numunutau{\nu_\mu\rightarrow\nu_\tau}
\def\nuebar{\bar\nu_e}
\def\nue{\nu_e}
\def\nutau{\nu_\tau}
\def\numubar{\bar\nu_\mu}
\def\numu{\nu_\mu}
\def\ra{\rightarrow}
\def\numubarnuebar{\bar\nu_\mu\rightarrow\bar\nu_e}
\def\nuebarnumubar{\bar\nu_e\rightarrow\bar\nu_\mu}
\def\osc{\rightsquigarrow}
\def\inteni{{\cal I}_{pot}}
\def\fmerit{{\cal F}}
%%%%%%%%%%%%%%%%%%%%%%%%%%%%%%%%%
\begin{flushright}
{\tt version 0.5}\\ 
\today
\end{flushright}
\vspace*{0.6cm}
%%%%%%%%%%%%%%%%%%%%%%%%%%%%%%%%%
%\linenumbers
%%%%%%%%%%%%%%%%%%%%%%%%%%%%%%%%%
\begin{center}
{\Large \bf Momentum based dE/dX Calibration of LArIAT for Run I and Run II} 
\vspace*{1.6cm}
\setcounter{footnote}{0}  
\def\A{\kern+.6ex\lower.42ex\hbox{$\scriptstyle \iota$}\kern-1.20ex a}
\def\E{\kern+.5ex\lower.42ex\hbox{$\scriptstyle \iota$}\kern-1.10ex e}
\small
\newcommand{\Aname}[2]{#1}
\def\titlefoot#1{\vspace{-0.3cm}\begin{center}{\bf #1}\end{center}}

\Aname{J.~Asaadi}{UTA},

\titlefoot{University of Texas at Arlington\label{UTA}}
%\vspace{-0.1cm}

\Aname{E. Gramellini}{Yale},

\titlefoot{Yale University\label{Yale}}
%\vspace{-0.1cm}

\Aname{G. Pullium}{Syracuse},

\titlefoot{Syracuse University\label{Syracuse}}
%\vspace{-0.1cm}

\end{center}
\vspace*{1cm}


%%%%%%%%%%%%%%%%%%%%%%%%%%%%%%%%%
%% ABSTRACT
%%%%%%%%%%%%%%%%%%%%%%%%%%%%%%%%%
%\newpage
\begin{abstract}

We present the calibration technique of parameterizing the energy deposited per unit length (dE/dX) as a function of momentum used to calibrate both the data and Monte Carlo samples for Run-I and Run-II. This calibration is done for a sample of $\pi, \mu$, and protons from the beamline and cross-checked using a sample of cosmic rays. The results for this calibration along with the derived calibration constants are presented.

\end{abstract} 

%%%%%%%%%%%%%%%%%%%%%%%%%%%%%%%%%
%% Table of content
%%%%%%%%%%%%%%%%%%%%%%%%%%%%%%%%%
\tableofcontents


%%%%%%%%%%%%%%%%%%%%%%%%%%%%%%%%%
%% SECTION 1: Introduction
%%%%%%%%%%%%%%%%%%%%%%%%%%%%%%%%%
\newpage
\section{Introduction}\label{sec:Introduction}

Here we present the results of the calorimetry tuning done for a sample of data and Monte Carlo spanning Run I and Run II of the LArIAT experiment. This calibration method is predicated on the Bethe-Bloch description of the mean rate of energy loss for various particle species. This is best represented by Figure \ref{fig:PDGEnergyLoss}, taken from the Particle Data Group \cite{PDG}.

\begin{figure}[htb]
\centering
\includegraphics[width=0.50\textwidth]{images/PDGdEdX.png}
\caption{Mean energy loss in various materials over a range of particle momentums as produced in Reference \cite{PDG}.}
\label{fig:PDGEnergyLoss}
\end{figure}

Using the tables provided by the PDG for liquid argon (\cite{PDG-Argon}), we calculate the theoretical values for pions ($\pi$), muons ($\mu$), kaons ($k$) and protons ($p$) in the momentum range most relevant for LArIAT, shown in Figure \ref{fig:PDGEnergyLossArgon}. This prediction utilizes the Bethe-Bloch description of how much energy charged particles deposit corrected for the density using the Sternheimer parameterization. The most probable value is calculated as

\begin{equation}
\Delta_{p} (MPV) = \eta [ln (\frac{2 m_{e} c^2 \beta^{2} \gamma^{2}}{I}) + ln(\frac{\eta}{I}) + j - \beta^{2} - \delta(\beta \gamma)]
\end{equation}

where $\eta = \frac{1}{2} K z^2 \frac{Z x}{A \beta^2}$. Plotting $\Delta{p} / x$ shows the ``Fermi-Plateau'' for the various particles. A more in depth treatment of this function is given in \href{https://lartpc-docdb.fnal.gov/cgi-bin/private/RetrieveFile?docid=2567&filename=30June2017_Presentation_corrected.pdf&version=4}{docDB-2567}

\begin{figure}[htb]
\centering
\includegraphics[width=0.60\textwidth]{images/dEdXvsMomentumTemplate}
\caption{Mean energy loss (solid line) and most probable value (dashed line) for pions, muons, kaons, and protons in liquid argon over the momentum range most relevant for LArIAT.}
\label{fig:PDGEnergyLossArgon}
\end{figure}

Using the predictions in Figure \ref{fig:PDGEnergyLossArgon}, allows us to tune the calorimetry constants used to convert the ADC to charge. The goal is to have the data and MC agree across the broad range of momentum. This tuning is done in addition to the wire-by-wire corrections (described in detail in \href{http://lartpc-docdb.fnal.gov:8080/cgi-bin/RetrieveFile?docid=1994&filename=investigation-uniformity-observed_v3.pdf&version=2}{doc-DB 1994}) and the usual lifetime corrections (described in detail in \href{http://lartpc-docdb.fnal.gov:8080/cgi-bin/ShowDocument?docid=1804}{docDB-1804}) which are used here and a more detailed treatment is left to their corresponding technical notes.

%%%%%%%%%%%%%%%%%%%%%%%%%%%%%%%%%%%%%%%%%%%%%%%%%%%%%%%%%%%
\subsection{Calibration Method Overview}\label{sec:MethodOverview}
%%%%%%%%%%%%%%%%%%%%%%%%%%%%%%%%%%%%%%%%%%%%%%%%%%%%%%%%%%%

Here describe the methodology used to select our samples and tune the calorimetry constants. Details specific to any one sample will be described in Section  \ref{sec:EventSelection} and we present the general method here and how it is applied. 

The basic idea of this calibration technique is to utilize a portion of a track within the LArTPC that has a well known momentum and particle species to measure the energy deposited per unit length (dE/dX) as recorded inside the TPC. Once a sample of particles dE/dX has been measured at various momentums, we then tune to calorimetry constants within the reconstruction software to align these measured values to match the theoretical ones found in Figure \ref{fig:PDGEnergyLossArgon}.

Since various electronics changes were done between Run-I and Run-II data taking periods, we derive calibration constants independently for both of these run periods. These calibration constants are the factors which convert the charge collected (dQ) to energy (dE). The details of how the calorimetry package works is beyond the scope of this note and is given in \href{http://lartpc-docdb.fnal.gov:8080/cgi-bin/ShowDocument?docid=2444}{docDB-2444}.

The calibration procedure follows the following basic steps:
\begin{itemize}

\item \textbf{Particle identification in the beamline:}

We first select a sample of beamline events that correspond to either a sample of $\pi, \mu, e$ or protons. This is done by selecting based on the measured time-of-flight (TOF) and momentum. 

For the calibration samples we require the tracks reconstructed in the wire chamber satisfy the criteria known as a ``picky track''. ``Picky tracks'' correspond to tracks reconstructed using hits in all four wire chambers. In these events, one and only one hit in each wire chamber track can be reconstructed per event and the track satisfies a straightness requirement in the Y-Z plane. These tracks have a more accurate measure of the particle momentum than the ``high yield'' tracks which only require hits in three out of four of the wire chamber tracks and can have mulitple wire chamber tracks reconstructed per event. Details about wire chamber track reconstruction can be found in \cite{WCTrackReco}

The wire chamber track is extrapolated to the front face of the LArTPC giving a $x, y$ position expected for the track in the TPC. A simple flat correction to the momentum as measured by the wire chambers due to the  energy loss by the particle as it traverses the material between the last wire chamber and the front face of the LArTPC is applied in data. 

For the Monte Carlo, no such beamline identification is done and instead we use the data-driven Monte Carlo (DDMC), which constructs the momentum and angular distributions for the single particle MC, and launches particles from the position of the fourth wire chamber (WC4) towards the TPC. We utilize the Monte Carlo truth for the position and momentum of the Monte Carlo particles as they enter the TPC.

\item \textbf{Matching LArTPC tracks to the wire chamber tracks:}

For each track reconstructed inside the TPC, the most upstream trajectory point (smallest $z$ position) is found. For that point we have it's $x, y, z$ as well as the calculated $p_{x}, p_{y}, and p_{z}$ (which is used to calculate the tracks $\theta, \phi$. Using this as well as the wire chamber tracks extrapolated $x, y, \theta, \phi$ we then select the event if there is one and only one TPC track which matches the wire chamber track. The exact matching criteria is given in Section \ref{sec:EventSelection}.

We now have the initial momentum (corrected for energy loss due to the upstream material) of the TPC track which we can use for our calibration. For the Monte Carlo, the ``wire chamber'' matching is done treating the true position of the MC particle as it enters the TPC as the ``wire chamber'' track. The rest of the matching proceeds just as before.

\item \textbf{dE/dX sampling:}

With the track within the TPC identified and the momentum for that track measured, we require the track to be of a minimum length of 10~cm long (to ensure we are away from any interaction point where the track may be broken into subsequent tracks). We then take the first twelve spacepoints of the track (excluding the first point to avoid edge effects near the field cage) and sample the reconstructed dE/dX for each point along the track. On average, this samples 5~cm of the track (shown in Section \ref{sec:Results}). These dE/dX measurments are then put into a histogram that corresponds to measured momentum of the track.

The dE/dX histograms are sampled every 50 MeV in momentum (e.g. 150~MeV $< P <$ 200~MeV, 200~MeV $< P <$ 250~MeV, etc...). On average, pions and muons only lose $\sim$10 MeV in this 5~cm section of the track and protons lose $\sim$20 MeV. Thus choosing 50 MeV size bins for our histograms should more than cover the energy spread within those bins due to energy loss from ionization.

This process of selecting, sampling, and recording the dE/dX for various momentum bins is now repeated over the entire sample of events, allowing us to collect sufficient statistic in most of the momentum bins between 150~MeV and 1100~MeV (which varies slightly, depending on the initial momentum spectrum for a given run period).

\item \textbf{Fit, tune, repeat:}

Each 50 MeV momentum binned dE/dX histogram is now fit with a simple Landau function. The fit range for the pion/muon sample is between 1.0 MeV and 5.0 MeV (bounds chosen to avoid the low dE/dX fluctuations seen in Run-1) and 2.5 MeV and 8.0 MeV for the proton sample. The most probable value (MPV) and the associated error on the MPV from the fit are extracted and plotted on Figure \ref{fig:PDGEnergyLossArgon}.

Depending on the outcome of the fit, the calorimetry constants are either tuned up or down. As a rule of thumb, if the returned MPVs are too high (meaning the fitted dE/dX is above the line), the calorimetry constants are increased (as counter intuitive as this seems...it is how its done). This both shifts the mean as well as changes the shape (since the recombination factor is applied to the dE derived from the calorimetry constants). Likewise, if the values are low the calorimetry constants are decreased. 

The values are tuned for both the collection and induction plane to try to achieve the best match to the theoretical curve that we can. The exact match is left as a qualitative exercise and is not quantitatively evaluated. While the tuning of the induction plane was done, the results are not shown in the technical note because, for the purposes of the forthcoming analyses we will only use the collection plane.

In addition to these distributions, the cumulative dE/dX distribution is also plotted and fit with a Landau function to assess the overall dE/dX calibration for the sample. This procedure is illustrated in Figure \ref{fig:CalibrationExample}.

\end{itemize}

\begin{figure}[htb]
\centering
\includegraphics[width=0.50\textwidth]{images/CalibrationExample.png}
\caption{Illustration of the calibration technique. Here we depict a 325 MeV wire chamber track (shown in green) which enters the TPC (taking into account the energy loss from the upstream material) and we sample the first 12 spacepoints (shown in teal) to extract the dE/dX distribution which is fit with a Landau.}
\label{fig:CalibrationExample}
\end{figure}

With the procedure now laid out, we will move into the data and Monte Carlo samples used and the details of the event selection.


%%%%%%%%%%%%%%%%%%%%%%%%%%%%%%%%%
%% SECTION 2: Data Samples
%%%%%%%%%%%%%%%%%%%%%%%%%%%%%%%%%
\section{Data Samples}\label{sec:DataSamples}
The data samples used for this calibration span Run-I and Run-II. The details of all the data samples are outlined in \href{https://docs.google.com/spreadsheets/d/1_0kNCKBIIx53f6vopqN2OijtcTICHD9rDvN_YKGH2mI/edit?usp=sharing}{this data production spreadsheet} and summarized in Table \ref{tab:samples}. It is worth noting, that in the inclusive pion cross-section analysis (for which a good deal of this calibration is driven by), the ``High-Yield'' beamline reconstruction is what is used to increase the statistics of the sample. Since the ``Picky Track'' reconstruction used here represents only a subset of that data, we don't use the entire measurement sample to tune our response. 

The details of the beamline reconstruction and the mass filter are provided in the next section.

\begin{center}
\begin{table}[htb]
	\begin{center}
	%\resizebox{0.45\textwidth}{!}{%
	\begin{tabular}{c|c|c|c|c}
	\multicolumn{4}{c}{\textbf{Summary of Samples}} \\
	\hline \hline
	 Run Period & LArIATsoft Version & Beamline Reconstruction & Mass Filter & Calibration/Validation\\
	\hline
	Run-I & \verb!v06_34_01! & Negative Polarity Picky Track & $\pi, \mu, e$ & Calibration \\
	\hline
	Run-I & \verb!v06_34_01! & Positive Polarity Picky Track & $\pi, \mu, e$ & Validation \\
	\hline
	Run-I & \verb!v06_34_01! & Positive Polarity Picky Track & Proton & Validation \\
	\hline
	Run-II & \verb!v06_34_01! & Positive Polarity Picky Track & $\pi, \mu, e$ & Calibration \\
	\hline
	Run-II & \verb!v06_34_01! & Negative Polarity Picky Track & $\pi, \mu, e$ & Validation \\
	\hline
	Run-II & \verb!v06_34_01! & Positive Polarity Picky Track & Proton & Validation \\
	\hline
	\end{tabular}%}
	\caption{Summary of the data samples used for the calibration study. A sample listed as a ``calibration'' sample was used, tuned, checked, and re-tuned until the desired calibration was achieved. A sample listed at ``validation'' was then checked using the calorimetry constants derived for that run period. }
	\label{tab:samples}
	\end{center}
\end{table}
\end{center}

The corresponding data driven Monte Carlo (DDMC) samples which match the data samples listed in Table \ref{tab:samples} are used to tune the MC calibration constants and can be found on the data spreadsheet listed above.



\subsection{Event Selection}\label{sec:EventSelection}
For each sample, we outline the event selection used to select this data.

\begin{itemize}
\item \textbf{Time Stamp Filter}

A filter is used to select events which occured in time with the beam. These events typically coincide with the first 6 seconds of the beam spill, and therefore the events are filtered using the following LArIATsoft settings

\begin{verbatim}
tfilt:      @local::lariat_timestampfilter

# ====================================================================
# Specify range of events to select.  For Run I/II:
#   - pedestal events:  ~ 0.  - 1.2 sec
#   - beam events:      ~ 1.2 - 5.5 sec
#   - cosmic events:    ~ > 5.5 sec
#   (default selects ALL events)
physics.filters.tfilt.T1:                       1.2
physics.filters.tfilt.T2:                       5.5
physics.filters.tfilt.RequireRawDigits:         true

\end{verbatim}



\item \textbf{Beamline Reconstruction}

The standard LArIAT beamline reconstruction is used to select events which have a wire chamber track and TOF information in an individual event using the following modules.
\begin{verbatim}
### beamline elements ###

wctrack:     @local::lariat_wctrackbuilder
tof:         @local::lariat_tof
agcounter:   @local::lariat_aerogel
\end{verbatim}


\textbf{For Run-I we use these default parameters:}
\begin{verbatim} 
physics.producers.wctrack.PickyTracks:                          true
physics.producers.tof.HitThreshold:                           -10.0  
physics.producers.tof.HitDiffMeanUS:                            0.6  
physics.producers.tof.HitDiffMeanDS:                            1.0  
physics.producers.tof.HitMatchThresholdUS:                      3.0  
physics.producers.tof.HitMatchThresholdDS:                      6.0  
physics.producers.tof.HitWait:                                  20.
\end{verbatim}

\textbf{For Run-II we use these default parameters:}
\begin{verbatim} 
physics.producers.wctrack.PickyTracks:                          true
physics.producers.tof.HitThreshold:                             -3.
physics.producers.tof.HitDiffMeanUS:                            0.5  
physics.producers.tof.HitDiffMeanDS:                            0.4  
physics.producers.tof.HitMatchThresholdUS:                      3.0  
physics.producers.tof.HitMatchThresholdDS:                      6.0  
physics.producers.tof.HitWait:                                  20.
\end{verbatim}

\item \textbf{Particle Mass Filtering}
Using the beamline reconstruction, it is possible to calculate the mass of a given track using the following equation

\begin{equation}
mass = \frac{p}{c}\sqrt{(\frac{TOF \times c}{l})^2 -1}
\end{equation}
where $p$ represents the measured momentum from the wire chamber, $TOF$ represents the time-of-flight measured as the difference between the two time-of-flight paddles in the LArIAT beamline, $l$ is path length the particle traveled down the beamline, and $c$ represents the speed of light.

Using this it is possible to plot the mass of each track reconstructed in the particle beamline, as shown in Figure \ref{fig:mass}. The classification of events into the different samples follows:

\begin{itemize}

\item $\pi, \mu, e$: 0~MeV $<$ mass $<$ 350~MeV

\item kaon: 350~MeV $<$ mass $<$ 650~MeV

\item proton: 650~MeV $<$ mass $<$ 3000~MeV

\end{itemize}

\begin{figure}[htb]
\centering
\includegraphics[width=0.70\textwidth]{images/mass.png}
\caption{The mass plotted for a sample of Run-II events reconstructed in the beamline. The classification of the events into $\pi, \mu, e$, kaon, or proton is based on this distribution.}
\label{fig:mass}
\end{figure}

\item \textbf{LArTPC Reconstruction}

Finally, events are reconstructed inside the TPC and we select events that satisfy the following requirements

\begin{itemize}
\item At least one track reconstructed inside the TPC with length $>$~10~cm
\item The event has fewer than three tracks with a length less than 5~cm reconstructed (shower veto)
\item One and only one TPC track can be matched to a wire chamber track (-4.0~cm$< \Delta X <$ 6.0~cm, -5.0~cm$< \Delta Y <$ 5.0~cm, $\alpha < 10$ degrees)

\end{itemize}


\end{itemize}

Events passing all these selection requirements are used for the dE/dX calibration. The event reduction table for all these cuts is provided in Section \ref{sec:Results} for each relevant sub-sample as well as the results from the calorimetry constant tuning.



%%%%%%%%%%%%%%%%%%%%%%%%%%%%%%%%%
%% SECTION 3
%%%%%%%%%%%%%%%%%%%%%%%%%%%%%%%%%
\newpage
\section{Results}\label{sec:Results}




%%%%%%%%%%%%%%%%%%%%%%%%%%%%%%%%%%%%%%%%%%%%%%%%%%%%%%%
\subsection{Picky Tracks: $\pi, \mu, e$}\label{sec:PickyTrkPiMuE}
%%%%%%%%%%%%%%%%%%%%%%%%%%%%%%%%%%%%%%%%%%%%%%%%%%%%%%%



%%%%%%%%%%%%%%%%%%%%%%%%%%%%%%%%%
%% SECTION 4
%%%%%%%%%%%%%%%%%%%%%%%%%%%%%%%%%
%%\section{\textcolor{blue}{R\&D Strategy}}
%\section{Conclusions}\label{sec:Conclusion}

The calibration of LArIAT's data and Monte Carlo samples over the Run-I and Run-II periods using the Bethe-Block description of the mean rate of energy loss for various particle species allows us to calibrate the dE/dX response for various samples.

In this note, we showed the calibration using negative and positive polarity data for samples of $\pi, \mu, e$ and protons and cosmic rays as well using pion and proton Monte Carlo.

The calibration constants derived for the various run periods and Monte Carlo are summarized as:

\begin{itemize}
\item \textbf{Run-I}: \verb!physics.producers.calo.CaloAlg.CalAreaConstants: [0.032,0.058]!

\item \textbf{Run-II}: \verb!physics.producers.calo.CaloAlg.CalAreaConstants: [0.021,0.0490]!

\item \textbf{Monte Carlo}: \verb!physics.producers.calo.CaloAlg.CalAreaConstants: [0.094, 0.101]!
\end{itemize}


Figure \ref{fig:PionCompareDataAndMC} shows the direct comparison of the dE/dX distribution for Run-I DDMC and Run-I $\pi, \mu, e$ data. With the tuning of the calibration constants, they agree in both MPV and shape. 

\begin{figure}[htb]
\centering
\includegraphics[width=0.48\textwidth]{images/dEdXPionDataMCcompare.png}
\includegraphics[width=0.48\textwidth]{images/dEdXPionDataMCcompareLog.png}
\caption{Comparison of the dE/dX distributions for Pion DDMC and Run-I Negative Polarity $\pi, \mu, e$ data. The distributions have been area normalized, the left hand side is the linear y-axis scale while the right hand side is the log scale for the y-axis.}
\label{fig:PionCompareDataAndMC}
\end{figure}


\newpage
\section{Conclusions}\label{sec:Conclusion}

The calibration of LArIAT's data and Monte Carlo samples over the Run-I and Run-II periods using the Bethe-Block description of the mean rate of energy loss for various particle species allows us to calibrate the dE/dX response for various samples.

In this note, we showed the calibration using negative and positive polarity data for samples of $\pi, \mu, e$ and protons and cosmic rays as well using pion and proton Monte Carlo.

The calibration constants derived for the various run periods and Monte Carlo are summarized as:

\begin{itemize}
\item \textbf{Run-I}: \verb!physics.producers.calo.CaloAlg.CalAreaConstants: [0.032,0.058]!

\item \textbf{Run-II}: \verb!physics.producers.calo.CaloAlg.CalAreaConstants: [0.021,0.0490]!

\item \textbf{Monte Carlo}: \verb!physics.producers.calo.CaloAlg.CalAreaConstants: [0.094, 0.101]!
\end{itemize}


Figure \ref{fig:PionCompareDataAndMC} shows the direct comparison of the dE/dX distribution for Run-I DDMC and Run-I $\pi, \mu, e$ data. With the tuning of the calibration constants, they agree in both MPV and shape. 

\begin{figure}[htb]
\centering
\includegraphics[width=0.48\textwidth]{images/dEdXPionDataMCcompare.png}
\includegraphics[width=0.48\textwidth]{images/dEdXPionDataMCcompareLog.png}
\caption{Comparison of the dE/dX distributions for Pion DDMC and Run-I Negative Polarity $\pi, \mu, e$ data. The distributions have been area normalized, the left hand side is the linear y-axis scale while the right hand side is the log scale for the y-axis.}
\label{fig:PionCompareDataAndMC}
\end{figure}






\newpage
%%%%%%%%%%%%%%%%%%%%%%%%%%%%%%%%%
%%  BIBLIOGRAPHY	
%%%%%%%%%%%%%%%%%%%%%%%%%%%%%%%%%
%\begin{thebibliography}{99}
\footnotesize

%1
\bibitem{DUNE_CDR}
R. Acciarri {\it et al.}, ``Long-Baseline Neutrino Facility (LBNF) and Deep Underground Neutrino Experiment (DUNE) Conceptual Design Report Volume 1: The LBNF and DUNE Projects'', arXiv:1601.05471  [physics.ins-det].

%2
\bibitem{cenf}
CENF Project, \url{https://edms.cern.ch/nav/P:CERN-0000096725:V0/P:CERN-0000096728:V0/TAB3};M. Nessi, ``CERN Neutrino Platform'', ICFA Neutrino European Meeting, Paris, 8-10 January 2014. 

%3
\bibitem{lbnf@cern}
M. A. Leigui de Oliveria {\it et al.}, ``Expression of Interest for a Full-Scale Detector Engineering Test and Test Beam Calibration of a Single-Phase LAr TPC'',
CERN-SPSC-2014-027, SPSC-EOI-011.

%4
\bibitem{dual}
S. Murphy (on behalf of the WA105 collaboration),``The WA105-3x1x1 m3 dual phase LAr-TPC demonstrator'', (2016) arXiv:1611.05846v1 [physics.ins-det].

%5
\bibitem{nu_review}
C. Patrignani {\it et al.} (Particle Data Group), ``Neutrino Mass, Mixing, and Oscillation'', Chin. Phys. C, 40, 100001 (2016)

%6
\bibitem{LBNE}
C. Adams {\it et al.}, ``Scientific Opportunities with the Long-Baseline Neutrino Experiment'', arXiv:1307.7335 [hep-ex]; \url{http://lbne.fnal.gov}.

%7
\bibitem{1d}
C. Rubbia, ``The liquid argon time projection chamber: a new concept for neutrino detectors'', CERN-EP-INT-77-08 (1977).

%8
\bibitem{1e}
F. Arneodo {\it et al.}, ``The ICARUS Experiment, A Second-Generation Proton Decay Experiment and Neutrino Observatory at the Gran Sasso Laboratory'',arXiv:0103008 [hep-ex] (2001).

%9
\bibitem{1z}
P.~Aprili {\it et al.},
``The ICARUS experiment: A second-generation proton decay experiment and
neutrino observatory at the Gran Sasso laboratory. Cloning of T600 modules to
reach the design sensitive mass. (Addendum)'', 
LNGS-EXP 13/89 add.2/01, and CERN-SPSC-2002-027.

%10
\bibitem{nomad}
J. Altegoer {\it et al.}, ``The NOMAD experiment at the CERN SPS'', Nucl. Instrum. Meth. A 404 (1998) 96.

%11
\bibitem{microboone}
\url{http://www-microboone.fnal.gov}

%12
\bibitem{lar1nd}
C. Adams  {\it et al.}, ``LAr1-ND: Testing Neutrino Anomalies with Multiple LAr TPC Detectors at Fermilab'', arXiV:1309.7987 [physics.ins-det].

%13
\bibitem{1i}
M. Zeller {\it et al.}, ``Ionization signals from electrons and alpha-particles in mixtures of liquid Argon and Nitrogen - perspectives on protons for Gamma Resonant Nuclear 
Absorption applications'', JINST 5 (2010) 10009.

%14
\bibitem{1j}
B. Rossi {\it et al.}, ``A prototype liquid Argon Time Projection Chamber for the study of UV laser multi-photonic ionization'', JINST 4 (2009) 07011.

%15
\bibitem{1k}
A. Ereditato {\it et al.}, ``Study of ionization signals in a TPC filled with a mixture of liquid Argon and Nitrogen'', JINST 3 (2008) 10002.

%16
\bibitem{1l}
I. Badhrees {\it et al.}, ``Measurement of the two-photon absorption cross-section of liquid argon with a time projection chamber'', New J. Phys. 12 (2010) 113024.

%17
\bibitem{argontube1}
A. Ereditato {\it et al.}, ``Design and operation of ARGONTUBE: a 5 m long drift liquid argon TPC'', JINST 8 (2013) P07002.

%18
\bibitem{argontube2}
M. Zeller {\it et al.}, ``First measurements with ARGONTUBE, a 5m long drift Liquid Argon TPC'', Nucl. Instrum. Meth. A718 (2013) 454.

%19
\bibitem{argontube3}
I. Badhrees {\it et al.}, ``ARGONTUBE: An R\&D liquid Argon Time Projection Chamber'', JINST 7 (2012) C02011. 

%20
\bibitem{1m}
S. Amerio {\it et al.}, ``Design, construction and tests of the ICARUS T600 detector'', Nucl. Instrum. Meth. A 527 (2004) 329.

%21
\bibitem{Doke:nc}
T.~Doke,
``A Historical View On The R\&D For Liquid Rare Gas Detectors'',
Nucl.\ Instrum.\ Meth.\ A { 327} (1993) 113.

%22
\bibitem{Willis:1974gi}
W.~J.~Willis and V.~Radeka,
``Liquid Argon Ionization Chambers as Total Absorption Detectors'',
Nucl.\ Instrum.\ Meth.\  { 120} (1974) 221.

%23
\bibitem{Aprile:1985xz}
E.~Aprile, K.~L.~Giboni and C.~Rubbia,
``A Study of Ionization Electrons Drifting Large Distances in Liquid and Solid
Argon",
Nucl.\ Instrum.\ Meth.\ A { 241} 62 (1985).

%24
\bibitem{3tons}
P.~Benetti {\it et al.},
``A 3 Ton Liquid Argon Time Projection Chamber'', 
Nucl.\ Instrum.\ Meth.\ A { 332} (1993) 395;\\

%25
\bibitem{Cennini:ha}
P.~Cennini {\it et al.},
``Performance of a 3 Ton Liquid Argon Time Projection Chamber'',  
Nucl.\ Instrum.\ Meth.\ A { 345} (1994) 230.

%26
\bibitem{50lt}
F.~Arneodo {\it et al.},
``The ICARUS 50 l LAr TPC in the CERN neutrino beam'', 
arXiv:hep-ex/9812006.

%27
\bibitem{t600paper}
S.~Amerio {\it et al.},
``Design, construction and tests of the ICARUS T600 detector'',
Nucl. Instrum. Meth., A527 (2004) 329  and references therein.

%28
\bibitem{USeffort}
B.~Baller {\it et al.},
``Liquid Argon Time Projection Chamber Research and Development in the United States'', JINST 9 (2014) T05005;  arXiv:1307.8166v3.

%29
\bibitem{argoneut}
C.~Anderson {\it et al.},
``The ArgoNeuT Detector in the NuMI Low-Energy Beam Line at Fermilab'', JINST 7 (2012) P10019;
arXiV:1205.6747.

%30
\bibitem{icarusUSA}
M. Antonello {\it et al.}, ``ICARUS at FNAL'', arXiv:1312.7252 [physics.ins-det].
\url{http://www.fnal.gov/directorate/program\_planning/Jan2014PACPublic/ICARUS.pdf}

%31
\bibitem{ARAPUCA}
A.A. Machado and E. Segreto, ``ARAPUCA a new device for liquid argon scintillation light detection'', JINST 11 (2016) C02004.

%32
\bibitem{Martin_thesis}
M. Auger, ``New Micromegas based Readout techniques for Imaging in Time Projection Chambers'', PhD Thesis, Albert Einstein Center for Fundamental Physics, Universitat Bern.

%33
\bibitem{bnl-electronics}
V. Radeka {\it et al.}, iopscience.iop.org/1742-6596/308/1/012021/pdf/1742-6596\_308\_1\_012021.pdf

%34
\bibitem{topmetal}
M. An {\it et al.}, ``A Low-Noise CMOS Pixel Direct Charge Sensor, Topmetal-II'', arXiv:1509.08611v2 [physics.ins-det].

%35
\bibitem{bernHV}
A. Blatter {\it et al.}, ``Experimental study of electric breakdowns in liquid argon at centimeter scale'', JINST 9 (2014) P04006.%arXiv:1401.6693. 

%36
\bibitem{loi_old}
M. Auger {\it et al.}, ``ArgonCube: a novel, fully-modular approach for the realization of large-mass liquid argon TPC neutrino detectors'', (2015) CERN SPSC LoI 243.

%37
\bibitem{HVBD}
M. Auger {\it et al.}, ``On the Electric Breakdown in Liquid Argon at Centimeter Scale'', arXiv:512.05968v2 [physics.ins-det].

%38
\bibitem{CRT}
M. Auger {\it et al.}, ``Multi-channel front-end board for SiPM readout'', arXiv:1606.02290v1 [physics.ins-det].

%39
\bibitem{CRT2}
M. Auger {\it et al.}, ``A Novel Cosmic Ray Tagger System for Liquid Argon TPC Neutrino Detectors'', arXiv:1612.04614v1 [physics.ins-det].

%40
\bibitem{modtest}
D. Smargianaki and U. Kose ``Report for the test performed at BERN; The Module Box - Prototype '', CERN Project Document No:1728992 (2016)

%41
\bibitem{ehn1extension} EHN1 Hall Extension for Neutrino Detector R\&D Experiments, EDMS Document No. 1350076 v. 3.

%42
\bibitem{bernHV2}
M. Auger {\it et al.}, ``A method to suppress dielectric breakdowns in liquid argon ionization detectors for cathode to ground distances of several millimeters '', JINST 9 (2014) P07023.%arXiv:1406.3660. 
%43
\bibitem{wa105}
\url{http://laguna.ethz.ch:8080/Plone/wa105}

%44
\bibitem{wa105-2}
I. De Bonis {et al.}, ``Technical Design Report for Large-Scale Neutrino Detectors Prototyping and Phased Performance Assessment in View of a Longbaseline Oscillation Experiment'', CERN-SPSC-2014-013; SPSC-TDR-004.


\end{thebibliography}
\textbf{}
\begin{thebibliography}{99}
\footnotesize

%1
\bibitem{DUNE_CDR}
R. Acciarri {\it et al.}, ``Long-Baseline Neutrino Facility (LBNF) and Deep Underground Neutrino Experiment (DUNE) Conceptual Design Report Volume 1: The LBNF and DUNE Projects'', arXiv:1601.05471  [physics.ins-det].

%2
\bibitem{cenf}
CENF Project, \url{https://edms.cern.ch/nav/P:CERN-0000096725:V0/P:CERN-0000096728:V0/TAB3};M. Nessi, ``CERN Neutrino Platform'', ICFA Neutrino European Meeting, Paris, 8-10 January 2014. 

%3
\bibitem{lbnf@cern}
M. A. Leigui de Oliveria {\it et al.}, ``Expression of Interest for a Full-Scale Detector Engineering Test and Test Beam Calibration of a Single-Phase LAr TPC'',
CERN-SPSC-2014-027, SPSC-EOI-011.

%4
\bibitem{dual}
S. Murphy (on behalf of the WA105 collaboration),``The WA105-3x1x1 m3 dual phase LAr-TPC demonstrator'', (2016) arXiv:1611.05846v1 [physics.ins-det].

%5
\bibitem{nu_review}
C. Patrignani {\it et al.} (Particle Data Group), ``Neutrino Mass, Mixing, and Oscillation'', Chin. Phys. C, 40, 100001 (2016)

%6
\bibitem{LBNE}
C. Adams {\it et al.}, ``Scientific Opportunities with the Long-Baseline Neutrino Experiment'', arXiv:1307.7335 [hep-ex]; \url{http://lbne.fnal.gov}.

%7
\bibitem{1d}
C. Rubbia, ``The liquid argon time projection chamber: a new concept for neutrino detectors'', CERN-EP-INT-77-08 (1977).

%8
\bibitem{1e}
F. Arneodo {\it et al.}, ``The ICARUS Experiment, A Second-Generation Proton Decay Experiment and Neutrino Observatory at the Gran Sasso Laboratory'',arXiv:0103008 [hep-ex] (2001).

%9
\bibitem{1z}
P.~Aprili {\it et al.},
``The ICARUS experiment: A second-generation proton decay experiment and
neutrino observatory at the Gran Sasso laboratory. Cloning of T600 modules to
reach the design sensitive mass. (Addendum)'', 
LNGS-EXP 13/89 add.2/01, and CERN-SPSC-2002-027.

%10
\bibitem{nomad}
J. Altegoer {\it et al.}, ``The NOMAD experiment at the CERN SPS'', Nucl. Instrum. Meth. A 404 (1998) 96.

%11
\bibitem{microboone}
\url{http://www-microboone.fnal.gov}

%12
\bibitem{lar1nd}
C. Adams  {\it et al.}, ``LAr1-ND: Testing Neutrino Anomalies with Multiple LAr TPC Detectors at Fermilab'', arXiV:1309.7987 [physics.ins-det].

%13
\bibitem{1i}
M. Zeller {\it et al.}, ``Ionization signals from electrons and alpha-particles in mixtures of liquid Argon and Nitrogen - perspectives on protons for Gamma Resonant Nuclear 
Absorption applications'', JINST 5 (2010) 10009.

%14
\bibitem{1j}
B. Rossi {\it et al.}, ``A prototype liquid Argon Time Projection Chamber for the study of UV laser multi-photonic ionization'', JINST 4 (2009) 07011.

%15
\bibitem{1k}
A. Ereditato {\it et al.}, ``Study of ionization signals in a TPC filled with a mixture of liquid Argon and Nitrogen'', JINST 3 (2008) 10002.

%16
\bibitem{1l}
I. Badhrees {\it et al.}, ``Measurement of the two-photon absorption cross-section of liquid argon with a time projection chamber'', New J. Phys. 12 (2010) 113024.

%17
\bibitem{argontube1}
A. Ereditato {\it et al.}, ``Design and operation of ARGONTUBE: a 5 m long drift liquid argon TPC'', JINST 8 (2013) P07002.

%18
\bibitem{argontube2}
M. Zeller {\it et al.}, ``First measurements with ARGONTUBE, a 5m long drift Liquid Argon TPC'', Nucl. Instrum. Meth. A718 (2013) 454.

%19
\bibitem{argontube3}
I. Badhrees {\it et al.}, ``ARGONTUBE: An R\&D liquid Argon Time Projection Chamber'', JINST 7 (2012) C02011. 

%20
\bibitem{1m}
S. Amerio {\it et al.}, ``Design, construction and tests of the ICARUS T600 detector'', Nucl. Instrum. Meth. A 527 (2004) 329.

%21
\bibitem{Doke:nc}
T.~Doke,
``A Historical View On The R\&D For Liquid Rare Gas Detectors'',
Nucl.\ Instrum.\ Meth.\ A { 327} (1993) 113.

%22
\bibitem{Willis:1974gi}
W.~J.~Willis and V.~Radeka,
``Liquid Argon Ionization Chambers as Total Absorption Detectors'',
Nucl.\ Instrum.\ Meth.\  { 120} (1974) 221.

%23
\bibitem{Aprile:1985xz}
E.~Aprile, K.~L.~Giboni and C.~Rubbia,
``A Study of Ionization Electrons Drifting Large Distances in Liquid and Solid
Argon",
Nucl.\ Instrum.\ Meth.\ A { 241} 62 (1985).

%24
\bibitem{3tons}
P.~Benetti {\it et al.},
``A 3 Ton Liquid Argon Time Projection Chamber'', 
Nucl.\ Instrum.\ Meth.\ A { 332} (1993) 395;\\

%25
\bibitem{Cennini:ha}
P.~Cennini {\it et al.},
``Performance of a 3 Ton Liquid Argon Time Projection Chamber'',  
Nucl.\ Instrum.\ Meth.\ A { 345} (1994) 230.

%26
\bibitem{50lt}
F.~Arneodo {\it et al.},
``The ICARUS 50 l LAr TPC in the CERN neutrino beam'', 
arXiv:hep-ex/9812006.

%27
\bibitem{t600paper}
S.~Amerio {\it et al.},
``Design, construction and tests of the ICARUS T600 detector'',
Nucl. Instrum. Meth., A527 (2004) 329  and references therein.

%28
\bibitem{USeffort}
B.~Baller {\it et al.},
``Liquid Argon Time Projection Chamber Research and Development in the United States'', JINST 9 (2014) T05005;  arXiv:1307.8166v3.

%29
\bibitem{argoneut}
C.~Anderson {\it et al.},
``The ArgoNeuT Detector in the NuMI Low-Energy Beam Line at Fermilab'', JINST 7 (2012) P10019;
arXiV:1205.6747.

%30
\bibitem{icarusUSA}
M. Antonello {\it et al.}, ``ICARUS at FNAL'', arXiv:1312.7252 [physics.ins-det].
\url{http://www.fnal.gov/directorate/program\_planning/Jan2014PACPublic/ICARUS.pdf}

%31
\bibitem{ARAPUCA}
A.A. Machado and E. Segreto, ``ARAPUCA a new device for liquid argon scintillation light detection'', JINST 11 (2016) C02004.

%32
\bibitem{Martin_thesis}
M. Auger, ``New Micromegas based Readout techniques for Imaging in Time Projection Chambers'', PhD Thesis, Albert Einstein Center for Fundamental Physics, Universitat Bern.

%33
\bibitem{bnl-electronics}
V. Radeka {\it et al.}, iopscience.iop.org/1742-6596/308/1/012021/pdf/1742-6596\_308\_1\_012021.pdf

%34
\bibitem{topmetal}
M. An {\it et al.}, ``A Low-Noise CMOS Pixel Direct Charge Sensor, Topmetal-II'', arXiv:1509.08611v2 [physics.ins-det].

%35
\bibitem{bernHV}
A. Blatter {\it et al.}, ``Experimental study of electric breakdowns in liquid argon at centimeter scale'', JINST 9 (2014) P04006.%arXiv:1401.6693. 

%36
\bibitem{loi_old}
M. Auger {\it et al.}, ``ArgonCube: a novel, fully-modular approach for the realization of large-mass liquid argon TPC neutrino detectors'', (2015) CERN SPSC LoI 243.

%37
\bibitem{HVBD}
M. Auger {\it et al.}, ``On the Electric Breakdown in Liquid Argon at Centimeter Scale'', arXiv:512.05968v2 [physics.ins-det].

%38
\bibitem{CRT}
M. Auger {\it et al.}, ``Multi-channel front-end board for SiPM readout'', arXiv:1606.02290v1 [physics.ins-det].

%39
\bibitem{CRT2}
M. Auger {\it et al.}, ``A Novel Cosmic Ray Tagger System for Liquid Argon TPC Neutrino Detectors'', arXiv:1612.04614v1 [physics.ins-det].

%40
\bibitem{modtest}
D. Smargianaki and U. Kose ``Report for the test performed at BERN; The Module Box - Prototype '', CERN Project Document No:1728992 (2016)

%41
\bibitem{ehn1extension} EHN1 Hall Extension for Neutrino Detector R\&D Experiments, EDMS Document No. 1350076 v. 3.

%42
\bibitem{bernHV2}
M. Auger {\it et al.}, ``A method to suppress dielectric breakdowns in liquid argon ionization detectors for cathode to ground distances of several millimeters '', JINST 9 (2014) P07023.%arXiv:1406.3660. 
%43
\bibitem{wa105}
\url{http://laguna.ethz.ch:8080/Plone/wa105}

%44
\bibitem{wa105-2}
I. De Bonis {et al.}, ``Technical Design Report for Large-Scale Neutrino Detectors Prototyping and Phased Performance Assessment in View of a Longbaseline Oscillation Experiment'', CERN-SPSC-2014-013; SPSC-TDR-004.


\end{thebibliography}
\textbf{}



\end{document}